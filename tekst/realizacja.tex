\chapter{Hyperputer optymalizacji telechronicznej}

Ten rozdział powinien przedstawiać konkretne, autorskie osiągnięcia Dyplomantki lub Dyplomanty. Tytuł rozdziału i~podrozdziałów muszą być konkretne i~bezpośrednio powiązane z~tematem pracy dyplomowej. Ich liczba i~zawartość może się zmieniać w~zależności od tematyki pracy dyplomowej.

Między tytułem rozdziału a~podrozdziału musi być choćby krótki tekst. Może to być na przykład ogólne wprowadzenie czytelnika w~ten rozdział.

\section{Metodologia wykonania optymalizatora telechronicznego}

Metodologia czyli plan tego, w~jaki sposób i~z pomocą jakich narzędzi powinien zostać uzyskany oczekiwany rezultat, spełniający założenia celu pracy. Metodologia w~przypadku prac magisterskich powinna uwzględniać ,,element badawczy''.

\section{Konstrukcja i~oprogramowanie hyperputera}

Prezentacja konkretnych działań projektowych, technologicznych i~wykonawczych. To tu trzeba zaprezentować na przykład zaprojektowane urządzenie, obwód elektryczny albo napisane oprogramowanie. Ten tekst powinien być zilustrowany na przykład zdjęciami, schematami, listingami.

\section{Testy hyperputera w~próżni doskonałej}

Weryfikacja poprawności działania powstałego urządzenia lub oprogramowania. Opisy powinny opierać się o~dane zgromadzone w~tabelach i~zaprezentowane na wykresach.

Szczególnie w~tym podrozdziale praca magisterska powinna znacząco przewyższać złożonością prace inżynierskie poprzez zaprezentowanie i~omówienie wyników przeprowadzonych badań.
