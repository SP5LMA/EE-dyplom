\chapter{Hyperputer optymalizacji telechronicznej}

Ten rozdział powinien przedstawiać konkretne, autorskie osiągnięcia Dyplomantki lub Dyplomanty. Tytuły rozdziałów i podrozdziałów muszą być konkretne i bezpośrednio powiązane z tematem pracy dyplomowej.

\section{Metodologia wykonania optymalizatora telechronicznego}

Metodologia czyli plan tego, w jaki sposób i z pomocą jakich narzędzi zostanie uzyskany oczekiwany rezultat spełniający założenia celu pracy.

\section{Konstrukcja i oprogramowanie hyperputera}

Prezentacja konkretnych działań projektowych i wykonawczych.

\section{Testy hyperputera w próżni doskonałej}

Weryfikacja poprawności działania powstałego urządzenia lub oprogramowania.

Bardzo możliwe, że to w tym podrozdziale praca magisterska powinna znacząco przewyższać złożonością prace inżynierskie poprzez zaprezentowanie i omówienie wyników przeprowadzonych badań.
