\chapter{Telechronika stosowana}

Rozdział kolejny po ,,Wstępie'' zawiera podstawy teoretyczne, czyli informacje znane specjalistom i~dostępne w~literaturze przed rozpoczęciem realizacji danej pracy dyplomowej. Tytuł rozdziału i~podrozdziałów muszą być konkretne i~bezpośrednio powiązane z~tematem pracy dyplomowej.

Między tytułem rozdziału a~podrozdziału musi być choćby krótki tekst. Może to być na przykład ogólne wprowadzenie czytelnika w~ten rozdział.

Oto przykładowe odwołanie do literatury~\cite{fowler2009}. Chcąc odnieść się do kilku pozycji jednocześnie, trzeba ich znaczniki wpisać po przecinku~\cite{maxwell1865,leksinski1995}. Jeśli pozycji literaturowych podamy więcej to \LaTeX może połączyć je w~zakres tak jak w~tym przykładzie~\cite{fowler2009, maxwell1865, leksinski1995}. Można też odnieść się do konkretnej strony~\cite[s.~38]{leksinski1995}. ,,Przecięty'' w~połowie znacznik bibliograficzny nie wygląda dobrze. Poprawiłbym go zmieniając \underline{treść}, ale zostawiam tu dla przykładu.

Znak tyldy przed wywołaniem \texttt{cite} ,,skleja'' odnośnik z~poprzedzającym je słowem za pomocą tak zwanej ,,twardej'' spacji. Kropka kończąca zdanie znajduje się za odnośnikiem bibliograficznym. Wówczas jest to prawidłowe odwołanie do bibliografii.

Znaczniki takie jak ,,fowler2009'' i~,,maxwell1865'', które możesz zobaczyć w~źródłowym (.tex) pliku pracy, to identyfikatory pozycji bibliograficznych. Przekształceniem identyfikatorów na konkretne numery zajmuje się \LaTeX{}. Noty bibliograficzne mają określoną strukturę i~powinny znajdować się w~osobnym pliku, najlepiej o~nazwie zgodnej z~nazwą głównego pliku \TeX{owego} ale z~rozszerzeniem ,,bib'', czyli dla tego szablonu: \texttt{EE-dyplom.bib}. Plik ten jest plikiem tekstowym i~do jego modyfikacji nie potrzeba specjalistycznych narzędzi. Środowiska dedykowane edycji plików \LaTeX{owych}, takie jak \href{https://www.texstudio.org/}{TeXstudio} oraz \href{https://kile.sourceforge.io/}{Kile} mają wbudowane funkcje wspierające tworzenie not bibliograficznych.

Ponadto można wspomóc się aplikacjami do zarządzania bibliografią, takimi jak \href{https://www.jabref.org/}{JabRef}, \href{https://www.zotero.org/}{Zotero} albo \href{https://userbase.kde.org/KBibTeX}{KBibTex}. Są to programy otwartoźródłowe, wolne, darmowe i~wieloplatformowe przy czym Zotero zachęca do skorzystania z~opcjonalnego, odpłatnego przechowywania informacji w~chmurze. Alternatywnie można posłużyć się aplikacjami komercyjnymi: \href{https://www.mendeley.com/}{Mendeley} albo \href{https://endnote.com/}{Endnote}. Programów takich jest więcej a~ich przegląd znajduje się \href{https://en.wikipedia.org/wiki/Comparison_of_reference_management_software}{na Wikipedii}.

Ułatwieniem w~edycji pliku z~bibliografią może być to, że serwisy takie jak IEEE Xplore pozwalają na pobranie cytowania w~standardzie BibTeX. Mamy wówczas niemal pewność, że składnia w~pliku \texttt{EE-dyplom.bib} jest poprawna a~nota bibliograficzna zawiera komplet potrzebnych informacji. Inne serwisy gromadzące publikacje naukowe zazwyczaj mają podobne funkcje.

Każda zmiana pliku z~bibliografią wymaga jego rekompilacji. W~tym szablonie stosowany jest do tego program Biber. Zatem kompilacja odbywa się w~trzech krokach:
\begin{itemize}
	\item kompilacja pliku(-ów) *.tex -- rozpoznanie potrzebnych identyfikatorów not bibliograficznych za pomocą pdf\LaTeX{a},
	\item kompilacja pliku *.bib -- wydzielenie i~sformatowanie potrzebnych not bibliograficznych za pomocą Bibera,
	\item kompilacja pliku(-ów) *.tex -- podstawienie numerów pod identyfikatory i~dołączenie bibliografii za pomocą pdf\LaTeX{a}.
\end{itemize}
W~Overleaf i~wielu programach liczne kroki kompilacji są skonsolidowane i~mogą nawet nie być widoczne. Niniejszemu szablonowi towarzyszy prosty plik Makefile, który może stanowić sugestię kroków kompilacji ręcznej. Edytory takie jak TeXworks, TeXstudio, Kile zazwyczaj ,,wiedzą co mają robić'' lub czynią potrzebne narzędzia dostępniejszymi.

W tym szablonie noty bibliograficzne są sortowane w~kolejności ich przywołania w~tekście. Czyli w~treści ,,numerki'' pierwszych odwołań do bibliografii będą pojawiały się po kolei i~będą zgodne z~numeracją w~bibliografii. Dzieje się to automatycznie, również po zmianie kolejności przywołań literatury, więc nie trzeba się o~to martwić.

Co do zasady Biber umieści w~bibliografii tylko te noty bibliograficzne, które chociaż raz zostaną przywołane w~tekście. Dlatego w~pliku \texttt{EE-dyplom.bib} można przygotować sobie bibliografię ,,na zapas'' i~jednak z~niej nie skorzystać. Aczkolwiek należy pamiętać o~uruchomieniu Bibera przed ostateczną kompilacją pracy w~celu uporządkowania bibliografii. Jest poważnym błędem pozostawienie w~dokumencie takich not bibliograficznych, które nie są przywołane w~tekście.

\section{Podstawy psioniczne telechroniki}

Należy krótko opisać podstawy teoretyczne podejmowanego zagadnienia. W~tym celu trzeba powoływać się na literaturę -- książki, artykuły z~czasopism specjalistycznych i~branżowych oraz dokumenty techniczne takie jak patenty i~standardy.

Płatne serwisy udostępniające takie materiały są dostępne dla pracowników i~studentów PW dzięki \href{https://bg.pw.edu.pl/}{Bibliotece Głównej PW}, po zalogowaniu kontem bibliotecznym. Warte polecenia:

\begin{itemize}
	\item \href{http://eczyt.bg.pw.edu.pl/han/SpringerLink}{SpringerLink}
	\item \href{http://eczyt.bg.pw.edu.pl/han/Wiley}{Wiley Online Library}
	\item \href{http://eczyt.bg.pw.edu.pl/han/TaylorandFrancis}{Taylor and Francis Online}
\end{itemize}

Dodatkowe miejsca pozyskiwania wartościowych źródeł:

\begin{itemize}
		
	\item \href{https://uprp.gov.pl/}{Urząd Patentowy Rzeczypospolitej Polskiej}
	\item \href{https://www.epo.org/en}{European Patent Office}
	\item \href{https://www.uspto.gov/}{United States Patent and Trademark Office}
	\item \href{https://patents.google.com/}{Google Patents}
	\item \href{https://www.nist.gov/}{National Institute of Standards and Technology}
	\item \href{https://www.etsi.org/}{European Telecommunications Standards Institute}
\end{itemize}

\section{Przegląd stanu wiedzy}

Tytuł drugiego podrozdziału może pozostać taki, jak jest teraz. Przegląd stanu wiedzy to krytyczna analiza \underline{aktualnych} badań naukowych w~tematyce pracy dyplomowej. Aktualne publikacje naukowe to takie, które zostały opublikowane do kilku lub kilkunastu lat wstecz, w~zależności od zagadnienia. Ten fragment pracy dyplomowej powinien zostać oparty o~artykuły z~czasopism naukowych i~specjalistycznych. Polecane źródła:

\begin{itemize}
	\item \href{http://eczyt.bg.pw.edu.pl/han/ieee-import/}{IEEE}
	\item \href{http://eczyt.bg.pw.edu.pl/han/ScienceDirectOnLine}{Science Direct}
	\item \href{http://eczyt.bg.pw.edu.pl/han/ACMDigitalLibrary}{ACM}
	\item \href{https://pubmed.ncbi.nlm.nih.gov/}{NIH PubMed}
	\item \href{https://www.mdpi.com/}{MDPI}
	\item \href{https://arxiv.org/}{arXiv}
	\item \href{https://hal.science/}{Hyper Articles en Ligne}
\end{itemize}

Analiza literatury to nie jest zbiór przypadkowych ,,pozycji bibliograficznych'' z~wyszukiwarki internetowej! Musi ona:

\begin{itemize}
	\item ściśle wiązać się z~tematem pracy,
	\item poziomem złożoności odpowiadać zakresowi pracy dyplomowej, 
	\item kierować tekst pracy na drodze ,,od ogółu do szczegółu'', w~stronę rozdziału następnego, czyli do realizacji części ,,autorskiej''.
\end{itemize}

Praca magisterska ze względu na konieczny w~niej ,,wątek badawczy'' powinna bazować na szerszym zakresie literatury, która nie tylko zostanie omówiona ale również poddana krytycznej analizie.

Następna strona jest kompletnie pusta i~tak ma być, gdyż wymogiem edytorskim jest, by nowe rozdziały (przy druku dwustronnym) zaczynały się na stronie nieparzystej.
