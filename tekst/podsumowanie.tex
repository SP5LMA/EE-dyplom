\chapter{Podsumowanie}
\label{ch:podsumowanie}

Podsumowanie zazwyczaj nie potrzebuje podrozdziałów. Jednak gdyby osiągnęło około 5 stron, to można już zacząć planować jakiś podział tego rozdziału. Ponieważ podsumowanie odnosi się do zrealizowanej pracy i~uzyskanych wyników, to sensownym wydaje się napisać je w~czasie przeszłym. Poniżej sugestie odnośnie tego, co w~podsumowaniu powinno się znaleźć.

Akapit przypomnienia celu i~głównych założeń pracy. Jeśli w~trakcie realizacji pracy dyplomowej pewne wstępne cele szczegółowe lub założenia uległy weryfikacji i~być może subtelnym zmianom, to warto o~tym szczerze i~otwarcie napisać. Realizację pracy dyplomowej można sobie wyobrażać jako wyprawę do ściśle określonego celu ale po trasie, którą w~trakcie trwania podróży trzeba jednak urealniać, ze względu na napotkane warunki.

Podsumowanie i~dyskusja uzyskanych efektów. Nie powinno być to powtórzenie analizy rezultatów z~rozdziału poprzedniego a~raczej przedstawienie wniosków końcowych i~zwrócenie uwagi czytelnika na szczególnie ciekawe wyniki.

Odniesienie uzyskanych rezultatów do \underline{aktualnego} stanu wiedzy. Najlepiej, gdyby to odniesienie zostało poparte porównaniem konkretnych wyników uzyskanych w~toku realizacji pracy dyplomowej z~danymi prezentowanymi w~aktualnej literaturze -- wystarczy choćby jeden artykuł. Nie oznacza to, że trzeba uzyskać wyniki lepsze, niż prezentowane w~publikacjach naukowych.

Przedstawienie możliwości dalszego rozwoju podjętych działań i~badań. Te perspektywy można opisać w~kontekście trudności lub nawet niepowodzeń napotkanych podczas realizacji pracy dyplomowej. W~pracy magisterskiej można nieco ambitniej przedyskutować otwarte zagadnienia badawcze i~to, jak niniejsza praca dyplomowa może pomóc podjąć te wyzwania.

Można dodać akapit bardziej osobisty, dotyczący nowego doświadczenia zdobytego podczas realizacji pracy.

Jedno, końcowe zdanie subiektywnej oceny, w~jakim stopniu udało się zrealizować założony cel pracy i~spełnić postawione we wstępie wymagania (np.: częściowo, w~dużym stopniu, całkowicie).
