\chapter{Podsumowanie}
\label{ch:podsumowanie}

Podsumowanie zazwyczaj nie potrzebuje podrozdziałów. Jednak gdyby osiągnęło około 5 stron, to można już o takim jego podziale pomyśleć. Poniżej to, co w podsumowaniu powinno się znaleźć.

Akapit przypomnienia celu i głównych założeń pracy.

Podsumowanie i dyskusja uzyskanych efektów. Nie powinno być to powtórzenie analizy wyników z rozdziału poprzedniego a raczej przedstawienie wniosków końcowych i zwrócenie uwagi na szczególnie ciekawe rezultaty.

Odniesienie uzyskanych rezultatów do \underline{aktualnego} stanu wiedzy. Najlepiej, gdyby to odniesienie zostało poparte porównaniem konkretnych wyników uzyskanych w toku realizacji pracy dyplomowej z danymi prezentowanymi w aktualnej literaturze -- wystarczy choćby jeden artykuł.

Przedstawienie możliwości dalszego rozwoju podjętych działań i badań. W pracy magisterskiej można nieco ambitniej przedyskutować otwarte zagadnienia badawcze.

Można dodać akapit bardziej osobisty, dotyczący nowego doświadczenia zdobytego podczas realizacji pracy.

Jedno, końcowe zdanie subiektywnej oceny, w jakim stopniu udało się zrealizować założony cel pracy i spełnić postawione we wstępie wymagania (np.: częściowo, w dużym stopniu, całkowicie).